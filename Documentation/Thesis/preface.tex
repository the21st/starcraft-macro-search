\chapter{Introduction}
%\addcontentsline{toc}{chapter}{Introduction}

\section{Real-time Strategy Games}

One of the major genres on the video-game market are Real-time strategy (RTS) games. In a typical RTS, players build bases, command armies, and gather resources with the goal of destroying the opponents' units and buildings. As the words `real-time' suggest, the players can execute their actions simultanously and at any time, as opposed to turn-based strategies. As a result, RTS games are fast-paced and thus require quick decision making from the player. 

When talking about skills and techniques, professional RTS players use the terms micromanagement (micro) and macromanagement (macro), borrowed from the business sphere. Micro concerns local, short-term decisions and actions, such as commanding a single unit to attack a specific enemy unit or to move slightly in order to avoid damage from an explosion. Macro, on the other hand, encompasses long-term, global, strategic decisions, for example investing part of gathered resources into economic growth or ordering the player's army from defending her base to attack the enemy.

\section{Artificial intelligence in RTS games}



1. RTS games
2. existing AI in RTS games - commercial and research
3. My work - fill that hole
4. Outline of the thesis